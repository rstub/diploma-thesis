%%% Zusammenfassung und Ausblick
%%% Time-stamp: <1999-03-04 14:00:35 ralf>


\chapter{Zusammenfassung und Ausblick}
\label{cha:zusamm}

Im Rahmen dieser Arbeit wurde eine Verallgemeinerung der \kdotp-Theorie
abgeleitet, die eine effiziente Behandlung periodischer St"orungen erlaubt.
Dabei werden als Basis Bloch-Zust"ande des ungest"orten Systems verwendet, die
zu unterschiedlichen Punkten im reziproken Raum geh"oren, wobei die Auswahl
dieser Punkte durch die periodische St"orung bestimmt wird. Dadurch wird das
Problem auf eine algebraische Gleichung abgebildet, die neben den Parametern
der Standard \kdotp-Theorie noch zus"atzliche Potentialmatrixelemente
enth"alt.  Methoden, wie sie in der Standard \kdotp-Theorie verwendet werden,
k"onnen auf Grund der formalen "Ahnlichkeit zwischen den Gleichungen auf
unsere Fragestellung "ubertragen werden.

Als Beispiel einer Anwendung dieser Theorie diskutierten wir spontan geordnetes
\GaInP. Die in diesem Materialsystem beobachtete \CuPt-artige Ordnung f"uhrt
zu einer Wechselwirkung zwischen Zust"anden vom $\Gamma$- und L-Punkt der
ungeordneten Stuktur. Diese Wechselwirkung f"uhrt einerseits zu einer Mischung
der untersten Leitungsband-Zust"ande von $\Gamma$- und L-Punkt, was die
effektiven Masse im untersten Leitungsband erh"oht. Andererseits verkleinert
sie die Bandl"ucke, was eine Verringerung der effektiven Masse zur Folge hat.
Deshalb ist eine richtige Beschreibung der Abh"angigkeit der effektiven Masse
im Leitungsband vom Ordnungsgrad nur dann m"oglich, wenn diese Wechselwirkung
gut beschrieben wird.

Mit Hilfe der hier vorgestellten Theorie wurde ein Modell  entwickelt,
das sowohl die Wechselwirkung zwischen Zust"anden an verschiedenen Punkten der
Brillouin-Zone als auch die Ver"anderung der Wechselwirkung zwischen
Zust"anden am gleichen Punkt der Brillouin-Zone ber"ucksichtigt. Dies war
bisher im Rahmen von Standard \kdotp-Modellen nicht m"oglich. 

Die Form des Hamilton-Operators ist durch die Symmetrie der ungeordneten
Struktur und der ordnungsbedingten St"orung eingeschr"ankt. Mit Hilfe
gruppentheoretischer "Uberlegungen bestimmten wir diese Form und reduzierten
damit die Anzahl der in diesem Modell auftretenden Matrixelemente. Die
Betr"age dieser Matrixelemente erhielten wir aus Bandstrukturrechnungen oder
konnten sie an Ergebnisse anderer Untersuchungen anpassen. Der einzige freie
Parameter beschreibt den Grad an Ordnung. Die Bestimmung der Vorzeichen der
Potentialmatrixelemente erwies sich als schwierig. Wir konnten jedoch die
Anzahl der verschiedenen M"oglichkeiten auf zwei reduzieren, von denen eine
mit hoher Wahrscheinlichkeit auszuschlie"sen ist.

%Die Diagonalisierung des Hamilton-Operators nehmen wir in zwei Schritten
%vor. Im ersten Schritt diagonalisieren wir bez"uglich der ordnungsbedingten
%Potentialmatrixelemente und erhalten so und erhalten so neue Basisfunktionen,
%die sich als Linearkombination der alten Basisfunktionen schreiben lassen. In
%dieser neuen Basis enth"alt der Hamilton-Operators nur noch
%\kdotp-Wechselwirkungen, die wir in zweiter Ordnung St"orungstheorie
%behandeln, um die effektiven Massen zu erhalten. 

Die Ergebnisse f"ur die Ordnungsabh"angigkeit der effektiven Masse des
untersten Leitungsbandes stehen in guter "Ubereinstimmung mit
All-Elektronen-Rechnungen und experimentellen Daten. Au"serdem erhalten wir
Vorhersagen f"ur die effektiven Massen des r"uckgefalteten
Leitungsband-Zustandes des L-Punkts und die Impulsmatrixelemente zwischen
Leitungsband- und Va"-lenzband-Zust"anden. Unser Modell sagt anisotrope
Impulsmatrixelemente voraus. Diese Anisotropie ist wichtig f"ur die
Polarisationsabh"angigkeit optischer "Uberg"ange in diesem Material, wurde aber
bisher nicht ber"ucksichtigt.

F"ur zuk"unftige Untersuchungen erscheinen uns verschiedene Erweiterungen des
in dieser Arbeit verwendeten Modells interessant. So untersuchten wir nur die
Leitungsband-Zust"ande, doch l"a"st sich dieses Modell auch auf
Valenzband-Zust"ande anwenden, wenn aus Bandstrukturrechnungen die Parameter
bestimmt werden, welche die Dispersion im Valenzband des ungeordneten Systems
beschreiben. Auch die Einbeziehung der Spin-Bahn-Wechselwirkung sowie von
Verzerrungen ist m"oglich. Neben \GaInP\ k"onnen auch andere Materialien mit
\CuPt-atriger Ordnung beschrieben werden, doch w"are es dabei vorteilhaft die
Potentialmatrixelemente berechnen zu k"onnen, da nicht f"ur alle derartigen
Materialien gen"ugend Daten aus anderen Untersuchungen vorliegen. 

Neben \CuPt-Ordnung werden noch verschiedene andere Ordnungstypen in
III--V-Halbleiterlegierungen beobachtet. Solche Systeme k"onnen analog zu dem
hier beschriebenen Vorgehen behandelt werden, wenn die gruppentheoretische
Analyse entsprechend angepa"st wird. Eine weitere Anwendungs"-m"og"-lichkeit 
besteht in der Beschreibung k"unstlicher "Ubergitter, wo eine variable
St"orung durch unterschiedliche Periodizit"at erzeugt wird. Hier w"are es
sinnvoll zu untersuchen, ob Zust"ande von allen r"uckfaltenden Punkten des
reziproken Raumes ber"ucksichtigt werden m"ussen oder ob eine Beschr"ankung
auf ausgew"ahlt Punkte m"oglich ist. 




%%% Local Variables: 
%%% mode: latex
%%% TeX-master: "diplom"
%%% End: 
