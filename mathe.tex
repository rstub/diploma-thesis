% mathe.tex
% Time-stamp: <1999-09-23 19:34:49 ralf>
%
% Mathematisches allerlei
%
\DeclareMathAlphabet{\mathsfbf}{T1}{cmss}{bx}{n}
\DeclareMathAlphabet{\mathcalbf}{U}{eus}{b}{n}
%
% mathematische Objekte
%
\newcommand{\Matrix}[1]{\ensuremath{#1}}
% Operator (ohne Hut)
\newcommand{\op}[1]{\ensuremath{\protect #1}}
\newcommand{\vecop}[1]{\ensuremath{\mathbf{\protect #1}}}
\newcommand{\grvecop}[1]{\mbox{\mathversion{bold}$\protect #1$}}
%
\newcommand{\menge}[1]{\ensuremath{\mathcal{#1}}}
\newcommand{\raum}[1]{\ensuremath{\mathsf{#1}}}
\renewcommand{\vec}[1]{\ensuremath{\mathbf{#1}}}
\newcommand{\grvec}[1]{\mbox{\mathversion{bold}$#1$}}
% Einheitsvectoren
\newcommand{\ex}{\ensuremath{\vec{\protect\hat e}_{x}}}
\newcommand{\ey}{\ensuremath{\vec{\protect\hat e}_{y}}}
\newcommand{\ez}{\ensuremath{\vec{\protect\hat e}_{z}}}
%
% Operationen/Operatoren
%
% hermitesch konjugiert, transponiert, komplex konjugiert
\newcommand{\hc}[1]{\ensuremath{#1^{\dagger}}}
\newcommand{\transp}[1]{\ensuremath{#1^{\mathsf{T}}}}
\newcommand{\cc}[1]{\ensuremath{#1^{\ast}}}
% Skalarprodukt <#1, #2> und vec.vec 
\newcommand{\scp}[2]{\ensuremath{\left\langle {#1}, {#2} \right\rangle}}
\newcommand{\sprod}[2]{\ensuremath{\vec{#1} \cdot \vec{#2} }}
% Kommutator [#1, #2]
\newcommand{\kom}[2]{\ensuremath{\left [\, {#1}, {#2} \, \right] }}
% Antikommutator {#1, #2}
\newcommand{\akom}[2]{\ensuremath{\left \{\, {#1}, {#2} \, \right \}}}
% Poissonklammern
\newcommand{\poisson}[2]{\akom{#1}{#2}}
% Ableitungen etc.
\newcommand{\dx}[1]{d#1\,}
\newcommand{\vdx}[1]{d^{3}\!\!\: #1\,}
\newcommand{\ddx}[1]{\frac{d}{d#1}}
\newcommand{\pdx}[1]{\frac{\partial}{\partial#1}}
% Benannte mathem. Objekte
\newcommand{\kronecker}[2]{\ensuremath{\delta_{#1\,#2}}}
\newcommand{\Rn}[1]{\ensuremath{\real^{#1}}}
% Konstanten
\newcommand{\eins}{\ensuremath{\mathsf{1}}}
\newcommand{\halb}{\ensuremath{\frac{1}{2}}}
\newcommand{\veps}{\varepsilon}
\newcommand{\vphi}{\varphi}
\newcommand{\element}{\,\in\,}
% bra und ket etc...
\newcommand{\bra}[1]{\ensuremath{\langle {#1}|}}
\newcommand{\ket}[1]{\ensuremath{| {#1} \rangle}}
\newcommand{\braket}[2]{\ensuremath{\langle {#1} | {#2} \rangle}}
\newcommand{\matrixel}[3]{\ensuremath{\langle {#1} | {#2} | {#3} \rangle}}
\newcommand{\expect}[1]{\ensuremath{\langle {#1} \rangle}}
% f�r gro�e Eintr�ge
\newcommand{\Bra}[1]{\ensuremath{\left\langle {#1} \right|}}
\newcommand{\Ket}[1]{\ensuremath{\left| {#1} \right\rangle}}
\newcommand{\BraKet}[2]{\ensuremath{\left\langle {#1} | {#2} \right\rangle}}
\newcommand{\Matrixel}[3]{\ensuremath{\left\langle {#1}%
\left| {#2} \right| {#3} \right\rangle}}
\newcommand{\Expect}[1]{\ensuremath{\left\langle {#1} \right\rangle}}

%%%%%%%%%%%%%%%%%%%%%%%%%%%%%%%%%%%%%%%%%%%%%%%%%%%%%%
% von Roland 

%  %  \newcommand {\arc}{\mbox{$\rm <\hspace{-0.6em}
%  %               \raise0.3ex\hbox{$\scriptscriptstyle )$}$}}

\newcommand{\vektor}[1]{\ensuremath{\underline{\bf #1}}}
%\newcommand {\PSI}{\ensuremath{\mathbf{\underline{\psi}}}}
%\newcommand {\Matrix}[1]{\ensuremath{\underline{\underline{\bf #1}}}}
%\newcommand {\ableit}{\ensuremath{\frac{1}{i}\partial_z}}
\newcommand{\kdotp}{\ensuremath{\bf k \cdot p}}
\newcommand{\kreuz}{\ensuremath{\times}}
\newcommand{\eps}{\ensuremath{\varepsilon}}
% waum gibt es hier fehler???
%  \newcommand{\epsfett}[0][]{\mbox{\mathversion{bold}$#1 \varepsilon$}}
%  \newcommand{\pifett}[0][]{\mbox{\mathversion{bold}$#1 \pi$}}
%  \newcommand{\sigmafett}[0][]{\mbox{\mathversion{bold}$#1 \sigma$}}
%  \newcommand{\rhofett}[0][]{\mbox{\mathversion{bold}$#1 \sigmapi$}}
\newcommand{\BETA}[2]{\ensuremath{\displaystyle \frac{\hbar^2 #1}{2 m #2}}}
\newcommand{\spur}{{\mbox{tr}\hspace{0.3ex}}}
\newcommand{\kk}{\ensuremath{{\bf k}}}
\newcommand{\pp}{\ensuremath{{\bf p}}}
\newcommand{\rr}{\ensuremath{{\bf r}}}
\newcommand{\Qq}{\ensuremath{{\bf Q}}}
\newcommand{\dd}{{\rm d}}
\newcommand{\Ee}{\ensuremath{\mathcal E}}
\newcommand{\EE}{\mbox{\mathversion{bold}$\mathcal E$}}
\newcommand{\EEk}{\mbox{\small\mathversion{bold}$\cal E$}}
\newcommand{\ee}{{\rm e}}

\newcommand{\natur}{\mbox{\rm I\hspace{-0.12em}N}}
\newcommand{\real}{\mbox{\rm I\hspace{-0.12em}R}}
\newcommand{\komplex}{\mbox{\rm%
\hspace{0.7ex}\rule[0.15ex]{0.10ex}{1.35ex}\hspace{-0.7ex}C}}

\newcommand{\rational}{\mbox{\rm%
\hspace{0.7ex}\rule[0.15ex]{0.10ex}{1.35ex}\hspace{-0.7ex}Q}}

\newcommand{\indizes}[1]{{\mbox{\scriptsize #1}}}

\newcommand{\Ds}{\displaystyle}
\newcommand{\Ts}{\textstyle}
\newcommand{\Ss}{\scriptstyle}
\newcommand{\SSs}{\scriptscriptstyle}

% \newcommand {\bkopf}[1]{\refstepcounter{figure}\label{#1}Fig.\ \ref{#1}:}

% \newcounter{eintrag}
% \newenvironment{liste}[1]{\begin{list}{#1}{\usecounter{eintrag}%
%    \labelwidth1em \leftmargin1.5em \labelsep0.5em \itemsep0ex
%    \rightmargin0pt \topsep-\parskip \addtolength{\topsep}{1ex}%
%    \partopsep1ex \parsep0.6ex}}{\end{list}}

% \newenvironment{einrueck}[1]{\begin{list}{}{%
%    \settowidth{\labelwidth}{#1}%              Argument=Textstring
%    \setlength{\leftmargin}{\labelwidth}%
%    \addtolength{\leftmargin}{\labelsep}%
%    \parsep0.5ex plus0.2ex minus 0.2ex
%    \itemsep0.3ex
%    \renewcommand{\makelabel}[1]{##1\hfill}}}{\end{list}}

% \newenvironment{Einrueck}[1]{\begin{list}{}{%
%    \labelwidth#1%                              Argument=Massangabe
%    \labelsep1em
%    \setlength{\leftmargin}{\labelwidth}%
%    \addtolength{\leftmargin}{\labelsep}%
%    \parsep0.5ex plus0.2ex minus 0.2ex
%    \itemsep0.3ex
%    \renewcommand{\makelabel}[1]{##1\hfill}}}{\end{list}}

%%%%%%%%%%%%%%%%%%%%%%%%%%%%%%%%%%%%%%%%%%%%%%%%%%%%%%%%%%%%%%%%%

% \def\lesssim{\raisebox{0pt}[1pt][0pt]{$
%    \begin{array}{c}{<}\\[-1.6ex]{\sim}\end{array}$}}
% \def\gtrsim{\raisebox{0pt}[1pt][0pt]{$
%    \begin{array}{c}{>}\\[-1.6ex]{\sim}\end{array}$}}

% % \newcolumntype {s}[1]{@{\hspace{#1}}}
% \def\openone{\leavevmode\hbox{\small1\kern-0.8ex\normalsize1}}

% \endinput

%%% Local Variables: 
%%% mode: latex
%%% TeX-master: t
%%% End: 
