%%% Kapitel "uber LCAO
%%% Time-stamp: <1999-03-04 14:08:42 ralf>


%\addcontentsline{toc}{chapter}{Anhang}%{57}

\chapter[Die LCAO-Methode]{Die Methode der Linearkombination atomarer Orbitale}
\label{cha:lcao}

%\addcontentsline{toc}{chapter}{Anhang4}


\section{Grundlagen der LCAO-Methode}
\label{sec:lcao-allgemein}

In der Methode der Linearkombination atomarer Orbitale 
[\emph{linear combination of atomic orbitals} (LCAO), oft auch
\emph{tight-binding approximation} (TBA) genannt] verwenden wir Bloch-Summen
%
\begin{displaymath}
  \Phi_{i \vec{k}} (\vec{r}) = \frac{1}{\sqrt{N}} \sum_{j} e^{i
  \sprod{k}{R_{\mathnormal{j}}}} \phi_{i} (\vec{r} - \vec{R_{\mathnormal{j}}})
\end{displaymath}
%
als Basis f"ur die Entwicklung der L"osung des Problems eines Elektrons in
einem periodischen Potential. Die $\phi_{i} (\vec{r} -
\vec{R_{\mathnormal{i}}})$ stellen dabei atomare Orbitale dar, die auf dem
Atom am Ort \vec{R_{\mathnormal{j}}} lokalisiert sind und durch den Index $i$
charakterisiert werden.  Die Summe l"auft "uber alle $N$ "aquivalenten Atome
des Kristalls. Die diskrete Translationssymmetrie wird durch den Wellenvektor
\vec{k} charakterisiert. F"ur jedes Atom in der Einheitszelle und alle
atomaren Wellenfunktionen k"onnen wir solch eine Blochsumme konstruieren, und
erhalten so eine Basis f"ur die Entwicklung der Wellenfunktionen. Eine L"osung
f"ur des Problems eines Elektrons in einem periodischen Potential l"a"st sich
dann als $\psi_{n\vec{k}}(\vec{r}) = \sum_{i} a_{ni} \Phi_{i \vec{k}}$
schreiben, wobei die Summe "uber alle Atome in der Einheitszelle und alle
atomaren Wellenfunktionen in der Basis geht.

Schreiben wir den Hamilton-Operator f"ur das periodische Potential in
dieser Basis, so erhalten wir viele Matrixelemente, die nur schwer zu
berechnen sind. Deshalb wird die LCAO-Methode seit Slater und Koster
\cite{slko:54} meist als Interpolationsmethode verwendet. Dabei wird die
Anzahl der Matrixelemente durch verschieden N"aherungen reduziert. Danach wird
der Wert dieser Matrixelemente an experimentelle oder aus anderen Rechnungen
bekannte Energieeigenwerte an Punkten hoher Symmetrie in der Brillouin-Zone
angepa"st. Ist dies geschehen, so k"onnen wir die Dispersion f"ur die
gesamte Brillouin-Zone berechnen. Typische N"aherungen sind, da"s wir
nur wenige Orbitale pro Atom in die Basis aufnehemen (hier $sp^{3}$
bzw. $sp^{3}d^{5}s^{\ast}$), Wechselwirkungen ab einer gewissen Entfernung
zwischen den Atomen vernachl"assigen (hier nur
N"achste-Nachbar-Wechselwirkungen) und die auftretenden Matrixelemente durch
eine Kombination 
sogenannter Zwei-Zentren-Integrale (\emph{two-center integrals})
n"ahern.\footnote{Diese sind zu den kovalenten Bindungen in der Chemie
  vergleichbar.} 
Diese Reduzierung der Anzahl der Parameter ist auch deshalb notwendig, um bei
den meist wenigen bekannten Energieeigenwerten ein aussagekr"aftiges Modell zu
erhalten.%\marginpar{Berechnung von Matrixelementen}

%%%%%%%%%%%%%%%%%%%%%%%%
\section{$sp^{3}$-Basis f"ur qualitative Aussagen}
\label{sec:sp3}

Wie Harrison \cite{harr:80} gezeigt hat, lassen sich viele Eigenschaften
tetraedrisch koordinierter Halbleiter schon in einer $sp^{3}$-Basis mit nur
N"achste-Nachbar-Wechselwirkungen qualitativ erkl"aren. Wir wollen ein
solches Modell hier benutzen, um die in Kap.~\ref{sec:phase} gezeigte Form
der Wellenfuntkionen zu erkl"aren.

Dazu verwenden wir wie in Kap.~\ref{sec:k.p-H} das symmetrieangepa"ste
Koordinatensystem $\ex \parallel [1 \bar{1} 0]$, $\ey \parallel [1 1 \bar{2}]$
und $\ez \parallel [111]$. W"ahlen wir das Anion in der Einheitszelle als
Ursprung, so sind die umgebenden vier Kationen an den Positionen
%
\begin{equationarray*}{rrrr}
  \vec{d_{1}} = &
\frac{\sqrt{3}a_{\text{latt}}}{4}\left( 
    \begin{array}[c]{r}
      0 \\ 0 \\ 1
    \end{array}
\right) &
%%%
  \vec{d_{2}} = &
\frac{a_{\text{latt}}}{4\sqrt{3}}\left( 
    \begin{array}[c]{r}
      2\sqrt{2} \\ 0 \\ -1
    \end{array}
\right) \\[4ex]
%%%%%%
  \vec{d_{3}} = &
\frac{a_{\text{latt}}}{4\sqrt{3}}\left( 
    \begin{array}[c]{r}
      -\sqrt{2} \\ \sqrt{6} \\ -1
    \end{array}
\right) &
%%%
  \vec{d_{4}} = &
\frac{a_{\text{latt}}}{4\sqrt{3}}\left( 
    \begin{array}[c]{r}
      -\sqrt{2} \\ -\sqrt{6} \\ -1
    \end{array}
\right)
\end{equationarray*}
%
zu finden. 

%%%%
%%%% sp3 TBA Hamiltonoperator
%%%%
%\setcounter{floateq}{\value{equation}}
\begin{sidewaysfloateqnum}
%\renewcommand{\arraystretch}{1.6}
  \begin{equation}
    \label{eq:tba-sp3}
    \left(\begin{array}{@{\hspace{-9ex}}r@{\hspace{6ex}}cccc}
%
& \raisebox{2.0em}[0ex][0ex]{\ket{s_{a}}} 
& \raisebox{2.0em}[0ex][0ex]{\ket{x_{a}}} 
& \raisebox{2.0em}[0ex][0ex]{\ket{y_{a}}} 
& \raisebox{2.0em}[0ex][0ex]{\ket{z_{a}}} \\[-2ex]
%%%%%%
\bra{s_{c}}
& ss\sigma (g_{1}+g_{2}) 
& - \frac{\sqrt{2}}{3} s_{c}p_{a}\sigma (2g_{2}-3g_{3})
& - \sqrt{\frac{2}{3}} s_{c}p_{a}\sigma g_{4}
& - s_{c}p_{a}\sigma (g_{1}-\frac{1}{3} g_{2})\\[3ex]
%%%%%%
\bra{x_{c}}
& \frac{\sqrt{2}}{3} s_{c}p_{a}\sigma (2g_{2}-3g_{3})
& \begin{array}[c]{c}
pp\pi(g_{1}+g_{2}) - \\[1ex]
\frac{2}{9} (pp\pi - pp\sigma) (4g_{2}-3g_{3})
\end{array}
& \frac{\sqrt{12}}{9} (pp\pi - pp\sigma) g_{4}
&  \frac{\sqrt{2}}{9} (pp\pi - pp\sigma) (2g_{2}-3g_{3})\\[3ex]
%%%%%
\bra{y_{c}}
& \sqrt{\frac{2}{3}} s_{c}p_{a}\sigma g_{4}
& \frac{\sqrt{12}}{9} (pp\pi - pp\sigma) g_{4} 
& \begin{array}[c]{c}
pp\pi(g_{1}+g_{2}) +\\[1ex]
\frac{2}{3} (pp\pi - pp\sigma) g_{3}
\end{array}
& \frac{\sqrt{6}}{9} (pp\pi - pp\sigma) g_{4}\\[3ex]
%%%%%
\bra{z_{c}}
& s_{c}p_{a}\sigma (g_{1}-\frac{1}{3} g_{2})
& \frac{\sqrt{2}}{9} (pp\pi - pp\sigma) (2g_{2}-3g_{3})
& \frac{\sqrt{6}}{9} (pp\pi - pp\sigma) g_{4}
& \begin{array}[c]{c}
pp\sigma (g_{1}+g_{2}) + \\[1ex]
\frac{8}{9} (pp\pi - pp\sigma) g_{2}
\end{array}
    \end{array}\right)
  \end{equation}
  \begin{eqnarray*}
    g_{1} &=& e^{i \sprod{k}{d_{1}}}\\
    g_{2} &=& e^{i \sprod{k}{d_{2}}} + e^{i \sprod{k}{d_{3}}} 
            + e^{i \sprod{k}{d_{4}}} \\
    g_{3} &=& e^{i \sprod{k}{d_{3}}} + e^{i \sprod{k}{d_{4}}}\\  
    g_{4} &=& e^{i \sprod{k}{d_{3}}} - e^{i \sprod{k}{d_{4}}}\\
  \end{eqnarray*}
%\renewcommand{\arraystretch}{0.625}
\caption{Wechselwirkungsmatrix $V_{ac}$ zwischen atomaren Orbitalen auf Anion
  und Kation in einem Zinkblendegitter mit $\ex \parallel [1 \bar{1} 0]$, $\ey
  \parallel [1 1 \bar{2}]$ und $\ez \parallel [111]$ und $sp^{3}$-Basis.}
\end{sidewaysfloateqnum}



Bei Slater und Koster \cite{slko:54} sind die Beziehungen zwischen
Matrixelementen des Hamilton-Operators bez"uglich atomarer Wellenfunktionen
$\phi_{i}$ und den Zwei-Zentren-Integralen angegeben. Damit k"onnen wir
die Form des TBA-Hamilton-Operators
%
\begin{equation}
  \label{eq:tba-H}
  \op{H}_{0}^{\SSs \text{TBA}} = \left(
  \begin{array}[c]{cc}
    E_{c}       & V_{ac} \\
    \hc{V_{ac}} & E_{a}
  \end{array} \right)
\end{equation}
%
bestimmen. $E_{c}$ und $E_{a}$ sind Diagonalmatrixen mit den sogenannten
\emph{on-site} Energien der verschiedenen atomaren Orbitalen auf der
Hauptdiagonalen. 
%
\begin{displaymath}
  E_{i} = \left(
  \begin{array}[c]{cccc}
E_{s}^{i} & 0 & 0 & 0 \\
0 & E_{p}^{i} & 0 & 0 \\
0 & 0 & E_{p}^{i} & 0 \\
0 & 0 & 0 & E_{p}^{i}
  \end{array} \right)
\end{displaymath}
%
$V_{ac}$ ist die Wechselwirungsmatrix zwischen den Orbitalen
auf Anion und Kation. Sie ist in Gl.~\eqref{eq:tba-sp3} dargestellt.

F"ur den $\Gamma$-Punkt mit $k=0$ zerf"allt diese $8\times8$-Matrix in vier
$2\times2$-Matrizen. Eine davon liefert uns den \GCB-Zustand, d.~h.\ das obere
Niveau des Zwei-Niveau-Systems:
%
\begin{displaymath}
  \left(
    \begin{array}[c]{cc}
E_{s}^{c} & 4 ss\sigma \\
4 ss\sigma & E_{s}^{a}
    \end{array}
\right)
\end{displaymath}
%
Da $ss\sigma<0$ gilt \cite{harr:80}, ist das obere Niveau gegenphasig. Der
\GCB-Zustand ist also durch
%
\begin{displaymath}
  \ket{\GCB} = + \alpha_{\Gamma}^{\text{c}} \ket{s_{c}} -
  \beta_{\Gamma}^{\text{c}} \ket{s_{a}}
\end{displaymath}
%
gegeben, mit $\alpha_{\Gamma}^{\text{c}} > \beta_{\Gamma}^{\text{c}} >0$.

Die anderen drei Matrizen sind gleich und liefern das dreifach entarteten
\GVB-Niveau, d.~h.\ das untere Niveau des Zwei-Niveau-Systems:
%
\begin{displaymath}
  \left(
    \begin{array}[c]{cc}
E_{p}^{c} & \frac{4}{3} (pp\sigma + 2 pp\pi) \\
\frac{4}{3} (pp\sigma + 2 pp\pi) & E_{p}^{a}
    \end{array}
\right)
\end{displaymath}
%
Da $pp\sigma + 2 pp\pi>0$ gilt \cite{harr:80}, ist hier das untere Niveau
gegenphasig. Der \GVBx-Zustand ist also durch
%
\begin{displaymath}
  \ket{\GVBx} = + \alpha_{\Gamma}^{\text{v}} \ket{x_{c}} -
  \beta_{\Gamma}^{\text{v}} \ket{x_{a}}
\end{displaymath}
%
gegeben, mit $\beta_{\Gamma}^{\text{v}} > \alpha_{\Gamma}^{\text{v}} > 0$.
Analoges ergibt sich f"ur \GVBy\ und \GVBz.  Die Beziehungen zwischen den
$\beta_{\Gamma}^{\text{i}}$ und $\alpha_{\Gamma}^{\text{i}}$ gelten dabei,
weil das Anion im Vergleich zum Kation immer die geringere \emph{on-site}
Energie besitzt.

F"ur den L-Punkt mit $\vec{k} = 2\pi/a_{\text{latt}} (1/2,1/2,1/2)$ zerf"allt
unsere Matrix in zwei "aquivalente $2\times2$-Matrizen und eine
$4\times4$-Matrix. Die beiden $2\times2$-Matrizen liefern das zweifach
entartete \LVB-Niveau . Dabei ist das Au"serdiagonalelement eine komplexe 
Gr"o"se. Doch diese Phase hebt sich mit dem Phasenunterschied zwischen den
Wellenfunktionen auf Anion und Kation gerade auf, so da"s wir die
Wellenfunktion als Summe "uber atomare Orbitale mit reellen Koeffizienten
schreiben k"onnen.\footnote{Das entspricht der Aussage aus Kap.~\ref{sec:ME},
  da"s auch am L-Punkt die Wellenfuntionen reell gew"ahlt werden k"onnen.} 
Der \LVBx-Zustand ist also durch
%
\begin{displaymath}
  \ket{\LVBx} = + \alpha_{\text{L}}^{\text{v}} \ket{x_{c}} +
  \beta_{\text{L}}^{\text{v}} \ket{x_{a}}
\end{displaymath}
%
gegeben, mit $\beta_{\text{L}}^{\text{v}} > \alpha_{\text{L}}^{\text{v}} >
0$. Analoges ergibt sich f"ur \LVBy.

Auch in der verbleibenden $4\times4$-Matrix heben sich die verschiedenen
Phasen gegenseitig auf. Das \LCB-Niveau ist also der Zustand, der zum
zweith"ochs"-ten Eigenwerte dieser reellen $4\times4$-Matrix geh"ort. Dies ist
der Zustand mit gleichphasigen $s$-Orbitalen und gegenphasigen
$p_{z}$-Orbitalen, wenn wir die Zwei-Zentren-Integrale aus
Ref.~\cite{harr:80} verwenden:
%
\begin{displaymath}
  \ket{\LCB} = - \alpha_{\text{L}}^{\text{c}} \ket{s_{c}} 
               - \alpha_{\text{L}}^{\text{c}\prime} \ket{z_{c}} 
               -  \beta_{\text{L}}^{\text{c}} \ket{s_{a}} 
               +  \beta_{\text{L}}^{\text{c}\prime} \ket{z_{a}} 
\end{displaymath}

Der "Ubergang von der Einheitszelle des Zinkblende-Gitters zu der
Einheitszelle des CuPt-geordneten Gitters ist dann einfach. F"ur die
Wellenfunktionen am $\Gamma$-Punkt gilt, da"s die atomaren Orbitale auf allen
"aquivalenten Atomen gleiche Phase haben. Die Orbitale auf den Atomen A2 und
C2 aus Abb.~\ref{fig:wfkt-G}(a) haben also das gleiche Vorzeichen, wie die auf
A1 und C1. F"ur die Wellenfunktionen am L-Punkt, haben die Orbitale auf den
Atomen A2 und C2 gerade das umgekehrte Vorzeichen, wie die auf A1 und C1. Wir
erhalten also f"ur die Wellenfunktionen in der Einheitszelle der
CuPt-geordneten Struktur:
%
\begin{eqnarray*}
%%%%%
\ket{\GCB} &=& + \alpha_{\Gamma}^{\text{c}} (\ket{s_{c1}} + \ket{s_{c2}})
               -  \beta_{\Gamma}^{\text{c}} (\ket{s_{a1}} + \ket{s_{a2}}) \\
%%%%%
\ket{\GVBx}&=& + \alpha_{\Gamma}^{\text{v}} (\ket{x_{c1}} + \ket{x_{c2}})
               -  \beta_{\Gamma}^{\text{v}} (\ket{x_{a1}} + \ket{x_{a2}}) \\
%%%%%
\ket{\LCB} &=& - \alpha_{\text{L}}^{\text{c}} (\ket{s_{c1}} - \ket{s_{c2}}) 
               - \alpha_{\text{L}}^{\text{c}\prime} (\ket{z_{c1}} - \ket{z_{c2}}) \\
           &&  -  \beta_{\text{L}}^{\text{c}} (\ket{s_{a1}} - \ket{s_{a2}}) 
               +  \beta_{\text{L}}^{\text{c}\prime} (\ket{z_{a1}} - \ket{z_{a2}}) \\
%%%%%
\ket{\LVBx}&=& + \alpha_{\text{L}}^{\text{v}} (\ket{x_{c1}} - \ket{x_{c2}})
               +  \beta_{\text{L}}^{\text{v}} (\ket{x_{a1}} - \ket{x_{a2}})
%%%%%
\end{eqnarray*}
%
Dies entspricht den Wellenfunktionen, wie sie in Abb.~\ref{fig:wfkt-G} und
\ref{fig:wfkt-L} zu sehen sind.

%%%%%%%%%%%%%%%%%%%%%%%%%%%%%
\section{$sp^{3}d^{5}s^{\ast}$-Basis f"ur quantitative Aussagen}
\label{sec:sp3d5s*}

TBA-Modelle mit einer $sp^{3}$-Basis sind gut geeignet um qualitative Aussagen
zu treffen. Doch insbesondere f"ur quantitative Aussagen "uber die
Leitungsb"ander, wie wir sie ben"otigen um die Eigenergien der Basiszust"ande
in unserem \kdotp-Modell und die Impulsmatrixelemente festzulegen, sind sie
weniger geeignet.\footnote{F"ur eine Diskussion der Probleme sie
  Ref.~\cite{jsbb:98}.}  
Jancu \emph{et al.} \cite{jsbb:98} haben f"ur verschiedene Gruppe-IV- und
III-V-Halbleiter empirische TBA-Parameter f"ur eine
$sp^{3}d^{5}s^{\ast}$-Basis bestimmt, und dabei eine gute Beschreibung der
Valenzb"ander und der untersten beiden Leitungsb"ander erhalten. Diese
Parametrisierung wollen wir hier als Ausgangspunkt verwenden.

Der TBA-Hamilton-Operator hat wieder die Form \eqref{eq:tba-H}, nur da"s
$E_{i}$ und $V_{ac}$ $10\times10$-Matrizen sind. Die Form von $V_{ac}$ k"onnen
wir wieder mit Hilfe der Beziehungen zwischen den Matrixelementen des
Hamilton-Operators bez"uglich atomarer Wellenfunktionen $\phi_{i}$ und den
Zwei-Zentren-Integralen aus Ref.~\cite{slko:54} bestimmen. $V_{ac}$ ist in
Gl.~\eqref{eq:tba-sp3d5s} dargestellt.

%%%% Floats f�r LCAO Anhang
%%%% Time-stamp: <1999-03-03 23:54:56 ralf>


%%%%
%%%% Abk�rzungen f�r sp3d5s* TBA Hamiltonian
%%%%
\begin{floateqnum}
  \label{eq:abk-spds-tba}
   \begin{equationarray*}{*{4}{r@{=}l}}
%%%% g1, g2
     g_{1} & \multicolumn{3}{l}
     {e^{i \sprod{k}{d_{1}}} +  e^{i \sprod{k}{d_{2}}} +
      e^{i \sprod{k}{d_{3}}} + e^{i \sprod{k}{d_{4}}}} & 
     g_{2} & \multicolumn{3}{l}
     {e^{i \sprod{k}{d_{1}}} +  e^{i \sprod{k}{d_{2}}} -
      e^{i \sprod{k}{d_{3}}} - e^{i \sprod{k}{d_{4}}}} \\[2ex]
%%%% g3, g4
     g_{3} & \multicolumn{3}{l}
     {e^{i \sprod{k}{d_{1}}} -  e^{i \sprod{k}{d_{2}}} +
      e^{i \sprod{k}{d_{3}}} - e^{i \sprod{k}{d_{4}}}} &
     g_{4} & \multicolumn{3}{l}
     {e^{i \sprod{k}{d_{1}}} -  e^{i \sprod{k}{d_{2}}} -
      e^{i \sprod{k}{d_{3}}} + e^{i \sprod{k}{d_{4}}}} \\[3ex]
%%%% ss
    E_{ss} & ss\sigma  &  
    E_{s^{\ast}s} & s_{a}^{\ast}s_{c}\sigma  &
    E_{ss^{\ast}} & s_{a}s_{c}^{\ast}\sigma  &
    E_{s^{\ast}s^{\ast}} & s^{\ast}s^{\ast}\sigma \\[2ex]
%%%% sp und sd    
    E_{ps} & \frac{1}{\sqrt{3}}s_{c}p_{a}\sigma &
    E_{sp} & \frac{1}{\sqrt{3}}s_{a}p_{c}\sigma &
    E_{ds} & \frac{1}{\sqrt{3}}s_{c}d_{a}\sigma &
    E_{sd} & \frac{1}{\sqrt{3}}s_{a}d_{c}\sigma \\[2ex]
%%%% s*p und s*d
    E_{ps^{\ast}} & \frac{1}{\sqrt{3}}s^{\ast}_{c}p_{a}\sigma &
    E_{s^{\ast}p} & \frac{1}{\sqrt{3}}s^{\ast}_{a}p_{c}\sigma &
    E_{ds^{\ast}} & \frac{1}{\sqrt{3}}s^{\ast}_{c}d_{a}\sigma &
    E_{s^{\ast}d} & \frac{1}{\sqrt{3}}s^{\ast}_{a}d_{c}\sigma \\[2ex]
%%%% pp
    E_{pp}^{xx}   & \multicolumn{3}{l}{\frac{1}{3}(pp\sigma + 2 pp\pi)} &
    E_{pp}^{xy}   & \multicolumn{3}{l}{\frac{1}{3}(pp\sigma -   pp\pi)} \\[2ex]
%%%% dd
    E_{dd}^{xx}   & \multicolumn{3}{l}
    {\frac{1}{3}(dd\sigma + \frac{2}{3} dd\pi + \frac{4}{3} dd\delta)} &
    E_{dd}^{xy}   & \multicolumn{3}{l}
    {\frac{1}{3}(dd\sigma - \frac{1}{3} dd\pi - \frac{2}{3} dd\delta)} \\[2ex]
%%%% dd
    E_{dd}^{33}   & \multicolumn{3}{l}{\frac{1}{3}(2 dd\pi +  dd\delta)} &
    E_{dd}^{53}   & \multicolumn{3}{l}
    {E_{dd}^{35} = \frac{1}{3}(dd\pi -  dd\delta)} \\[2ex]
%%%% pd
    E_{pd}^{xx}   & \multicolumn{3}{l}
    {\frac{1}{3}(p_{a}d_{c}\sigma - \frac{2}{\sqrt{3}} p_{a}d_{c}\pi)} &
    E_{pd}^{xy}   & \multicolumn{3}{l}
    {\frac{1}{3}(p_{a}d_{c}\sigma + \frac{1}{\sqrt{3}} p_{a}d_{c}\pi)} \\[2ex]
%%%% dp
    E_{dp}^{xx}   & \multicolumn{3}{l}
    {\frac{1}{3}(p_{c}d_{a}\sigma - \frac{2}{\sqrt{3}} p_{c}d_{a}\pi)} &
    E_{dp}^{xy}   & \multicolumn{3}{l}
    {\frac{1}{3}(p_{c}d_{a}\sigma + \frac{1}{\sqrt{3}} p_{c}d_{a}\pi)} \\[2ex]
%%%% pd und dp
    E_{pd}^{53}   & \multicolumn{3}{l}{\frac{1}{\sqrt{3}} p_{a}d_{c}\pi} &
    E_{dp}^{35}   & \multicolumn{3}{l}{\frac{1}{\sqrt{3}} p_{c}d_{a}\pi} 
  \end{equationarray*}
  \stepcounter{equation}
  \caption{Abk"urzungen f"ur den Wechselwirkunsmatrix \eqref{eq:tba-sp3d5s}}
\end{floateqnum}


%%%%
%%%% sp3d5s TBA Hamiltonoperator
%%%%
\begin{sidewaysfloateqnum}
  \begin{equation}
    \label{eq:tba-sp3d5s}
    \left(\begin{widearray}{@{\hspace{-17ex}}r@{\hspace{7ex}}cc|ccc|ccc|cc}
%
& \raisebox{2.0em}[0ex][0ex]{\ket{s_{a}}} 
& \raisebox{2.0em}[0ex][0ex]{\ket{s_{a}^{\ast}}} 
& \raisebox{2.0em}[0ex][0ex]{\ket{x_{a}}} 
& \raisebox{2.0em}[0ex][0ex]{\ket{y_{a}}} 
& \raisebox{2.0em}[0ex][0ex]{\ket{z_{a}}} 
& \raisebox{2.0em}[0ex][0ex]{\ket{yz_{a}}} 
& \raisebox{2.0em}[0ex][0ex]{\ket{zx_{a}}} 
& \raisebox{2.0em}[0ex][0ex]{\ket{xy_{a}}} 
& \raisebox{2.0em}[0ex][0ex]{\ket{(3z^{2}-r^{2})_{a}}} 
& \raisebox{2.0em}[0ex][0ex]{\ket{(x^{2}-y^{2})_{a}}} \\[-4ex]
%%%%%%%
\bra{s_{c}}
&   E_{ss} g_{0}       &   E_{s^{\ast}s} g_{0}
& - E_{ps} g_{1}       & - E_{ps}        g_{2}   & - E_{ps} g_{3}
&   E_{ds} g_{1}       &   E_{ds}        g_{2}   &   E_{ds} g_{3}
&   0                  &   0\\
%%%%%%%
\bra{s_{c}^{\ast}}
&   E_{ss^{\ast}} g_{0}  &   E_{s^{\ast}s^{\ast}} g_{0}
& - E_{ps^{\ast}} g_{1}  & - E_{ps^{\ast}}        g_{2} & - E_{ps^{\ast}} g_{3}
&   E_{ds^{\ast}} g_{1}  &   E_{ds^{\ast}}        g_{2} &   E_{ds^{\ast}} g_{3}
&   0                  &   0\\
%%%%%%%
\hline
\bra{x_{c}}
&   E_{sp}      g_{1}  &    E_{s^{\ast}p} g_{1}
&   E_{pp}^{xx} g_{0}  &    E_{pp}^{xy}   g_{3}  &   E_{pp}^{xy} g_{2}
& - E_{dp}^{xx} g_{0}  & -  E_{dp}^{xy}   g_{3}  & - E_{dp}^{xy} g_{2}
& \frac{1}{\sqrt{3}} E_{dp}^{35} g_{1} & -E_{dp}^{35} g_{1} \\
%%%%%%%
\bra{y_{c}}
&   E_{sp}      g_{2}  &    E_{s^{\ast}p} g_{2}
&   E_{pp}^{xy} g_{3}  &    E_{pp}^{xx}   g_{0}  &   E_{pp}^{xy} g_{1}
& - E_{dp}^{xy} g_{3}  & -  E_{dp}^{xx}   g_{0}  & - E_{dp}^{xy} g_{1}
& \frac{1}{\sqrt{3}} E_{dp}^{35} g_{2} &  E_{dp}^{35} g_{2} \\
%%%%%%%
\bra{z_{c}}
&   E_{sp}      g_{3}  &    E_{s^{\ast}p} g_{3}
&   E_{pp}^{xy} g_{2}  &    E_{pp}^{xy}   g_{1}  &   E_{pp}^{xx} g_{0}
& - E_{dp}^{xy} g_{2}  & -  E_{dp}^{xy}   g_{1}  & - E_{dp}^{xx} g_{0}
& - \frac{2}{\sqrt{3}} E_{dp}^{35} g_{3} & 0  \\
%%%%%%%
\hline
\bra{yz_{c}}
&   E_{sd}      g_{1}  &    E_{s^{\ast}d} g_{1}
&   E_{pd}^{xx} g_{0}  &    E_{pd}^{xy}   g_{3}  &   E_{pd}^{xy} g_{2}
&   E_{dd}^{xx} g_{0}  &    E_{dd}^{xy}   g_{3}  &   E_{dd}^{xy} g_{2}
& \frac{1}{\sqrt{3}} E_{dd}^{35} g_{1} & -E_{dd}^{35} g_{1} \\
%%%%%%%
\bra{zx_{c}}
&   E_{sd}      g_{2}  &    E_{s^{\ast}d} g_{2}
&   E_{pd}^{xy} g_{3}  &    E_{pd}^{xx}   g_{0}  &   E_{pd}^{xy} g_{1}
&   E_{dd}^{xy} g_{3}  &    E_{dd}^{xx}   g_{0}  &   E_{dd}^{xy} g_{1}
& \frac{1}{\sqrt{3}} E_{dd}^{35} g_{2} &  E_{dd}^{35} g_{2} \\
%%%%%%%
\bra{xy_{c}}
&   E_{sd}      g_{3}  &    E_{s^{\ast}d} g_{3}
&   E_{pd}^{xy} g_{2}  &    E_{pd}^{xy}   g_{1}  &   E_{pd}^{xx} g_{0}
&   E_{dd}^{xy} g_{2}  &    E_{dd}^{xy}   g_{1}  &   E_{dd}^{xx} g_{0}
& - \frac{2}{\sqrt{3}} E_{dd}^{35} g_{3} & 0  \\
\hline
%%%%%%%
\bra{(3z^{2}-r^{2})_{c}}
&   0   &   0
& - \frac{1}{\sqrt{3}} E_{pd}^{53} g_{1} 
& - \frac{1}{\sqrt{3}} E_{pd}^{53} g_{2}
&   \frac{2}{\sqrt{3}} E_{pd}^{53} g_{3}
&   \frac{1}{\sqrt{3}} E_{dd}^{53} g_{1} 
&   \frac{1}{\sqrt{3}} E_{dd}^{53} g_{2}
& - \frac{2}{\sqrt{3}} E_{dd}^{53} g_{3}
& E_{dd}^{33} g_{0}  &  0 \\
%%%%%%%
\bra{(x^{2}-y^{2})_{c}}
&   0   &   0
&    E_{pd}^{53} g_{1}  &  - E_{pd}^{53} g_{2}  &  0
&  - E_{dd}^{53} g_{1}  &    E_{dd}^{53} g_{2}  &  0
&    0     &    E_{dd}^{33} g_{0}   \\
    \end{widearray}\right)
   \end{equation}
   \caption{Wechselwirkungsmatrix $V_{ac}$ zwischen atomaren Orbitalen auf 
     Anion und Kation in einem Zinkblendegitter mit $\ex \parallel [100]$, 
     $\ey \parallel [010]$ und $\ez \parallel [001]$ und 
     $sp^{3}d^{5}s^{\ast}$-Basis. Auftretende Abk"urzungen sind in 
     Gl.~\eqref{eq:abk-spds-tba} angegeben.}
\end{sidewaysfloateqnum}



%\newpage
%\begin{longtable}{l|*{5}{r}} %insgesamt 19 Spalten
%  \label{tab:tba-parameter}
%  \endhead
%  \hline
%  \caption{TBA Prameter}
%  \endfoot
\begin{table}
\begin{minipage}{\textwidth}
  \renewcommand{\thefootnote}{\thempfootnote}
  \begin{tabular}{l@{\hspace{4ex}}*{5}{@{\hspace{3.5ex}}r}}
\hline \hline
\raisebox{0ex}[3ex][0.5ex]{Material} 
& GaP\footnote{Ref.~\cite{jsbb:98}} %\addtocounter{footnote}{-1}
& InP\footnotemark[\value{mpfootnote}]
& GaP-InP\footnote{Differenz der Werte f"ur GaP und InP}
& \GaInP\footnote{linear Interpoliert f"ur Ga:In $=$ 51:49}
& \GaInP\footnote{Fit bez"uglich der $ss\sigma$-Martrixelemente} \\ \hline
\raisebox{0ex}[2.5ex][0.5ex]{$a_{\text{latt}}$ (\AA)}   &     5.4509 &     5.8687 &    -0.4178 &     5.6556 &    5.6556 \\[1ex]
% $E_{\langle 100 \rangle}$ &     5.0624 &     4.3673 &     0.6951 &     4.7218 &    4.7218 \\[1ex]
 $E^{a}_{s}$        &    -5.3379 &    -5.3321 &    -0.0058 &    -5.3351 &   -5.3351 \\
 $E^{c}_{s}$        &    -0.4005 &     0.3339 &    -0.7344 &    -0.0406 &   -0.0406 \\
 $E^{a}_{p}$        &     3.3453 &     3.3447 &     0.0006 &     3.3450 &    3.3450 \\
 $E^{c}_{p}$        &     6.3844 &     6.4965 &    -0.1121 &     6.4393 &    6.4393 \\
 $E_{d}$            &    14.0431 &    12.7756 &     1.2675 &    13.4220 &   13.4220 \\
 $E_{s^{\ast}}$     &    20.3952 &    18.8738 &     1.5214 &    19.6497 &   19.6497 \\[1ex]
 $ss\sigma$         &    -1.7049 &    -1.4010 &    -0.3039 &    -1.5560 &   -1.5100 \\
 $s^{\ast}s^{\ast}\sigma$  &    -3.5704 &    -3.6898 &     0.1194 &    -3.6289 &   -3.6470 \\
 $s_{a}^{\ast}s_{c}\sigma$ &    -1.6034 &    -1.8450 &     0.2416 &    -1.7218 &   -1.7600 \\
 $s_{a}s_{c}^{\ast}\sigma$ &    -1.6358 &    -1.2867 &    -0.3491 &    -1.4647 &   -1.4100 \\[1ex]
 $s_{a}p_{c}\sigma$ &     2.8074 &     2.1660 &     0.6414 &     2.4931 &    2.4931 \\
 $s_{c}p_{a}\sigma$ &     2.9800 &     2.6440 &     0.3360 &     2.8154 &    2.8154 \\
 $s_{a}^{\ast}p_{c}\sigma$ &     2.3886 &     2.5652 &    -0.1766 &     2.4751 &    2.4751 \\
 $s_{c}^{\ast}p_{a}\sigma$ &     2.1482 &     2.0521 &     0.0961 &     2.1011 &    2.1011 \\[1ex]
 $s_{a}d_{c}\sigma$ &    -2.7840 &    -2.5559 &    -0.2281 &    -2.6722 &   -2.6722 \\
 $s_{c}d_{a}\sigma$ &    -2.3143 &    -2.2192 &    -0.0951 &    -2.2677 &   -2.2677 \\
 $s_{a}^{\ast}d_{c}\sigma$ &    -0.6426 &    -0.7912 &     0.1486 &    -0.7154 &   -0.7154 \\
 $s_{c}^{\ast}d_{a}\sigma$ &    -0.6589 &    -0.8166 &     0.1577 &    -0.7362 &   -0.7362 \\[1ex]
 $pp\sigma$         &     4.1988 &     4.0203 &     0.1785 &     4.1113 &    4.1113 \\
 $pp\pi$            &    -1.4340 &    -1.2807 &    -0.1533 &    -1.3589 &   -1.3589 \\[1ex]
 $p_{a}d_{c}\sigma$ &    -1.7911 &    -1.9239 &     0.1328 &    -1.8562 &   -1.8562 \\
 $p_{c}d_{a}\sigma$ &    -1.8106 &    -1.8851 &     0.0745 &    -1.8471 &   -1.8471 \\
 $p_{a}d_{c}\pi$    &     1.8574 &     1.5679 &     0.2895 &     1.7155 &    1.7155 \\
 $p_{c}d_{a}\pi$    &     2.1308 &     1.7763 &     0.3545 &     1.9571 &    1.9571 \\[1ex]
 $dd\sigma$         &    -1.2268 &    -1.2482 &     0.0214 &    -1.2373 &   -1.2373 \\
 $dd\pi$            &     2.2752 &     2.1487 &     0.1265 &     2.2132 &    2.2132 \\
 $dd\delta$         &    -2.0124 &    -1.6857 &    -0.3267 &    -1.8523 &   -1.8523  \\[1ex] \hline \hline
% $\Delta_{a}/3$     &     0.0301 &     0.0228 &     0.0073 &  & \\
% $\Delta_{c}/3$     &     0.0408 &     0.1124 &    -0.0716 &  & \\ \hline
%
%\end{longtable}
\end{tabular}
  \caption{TBA-Prameter f"ur $sp^{3}d^{5}s^{\ast}$-Basis. Die Gitterkonstante
 $a_{\text{latt}}$  gilt f"ur Raumtemperatur. Die anderen Paramter sind 
 tieftemperatur Werte, die in  eV angegeben sind.}
  \label{tab:tba-parameter}
\end{minipage}
\end{table}
 


%%% Local Variables: 
%%% mode: latex
%%% TeX-master: "diplom"
%%% fill-column: 150
%%% End: 



Das Anion w"ahlen wir dabei wieder als Ursprung des Koordinatensystems,
diesmal aber mit $\ex \parallel [100]$, $\ey \parallel [010]$ und $\ez
\parallel [001]$. Die umgebenden Kationen sind also an den Positionen 
%
\begin{equationarray*}{rrrr}
  \vec{d_{1}} = &
\frac{a_{\text{latt}}}{4}\left( 
    \begin{array}[c]{r}
      1 \\ 1 \\ 1
    \end{array}
\right) &
%%%
  \vec{d_{2}} = &
\frac{a_{\text{latt}}}{4}\left( 
    \begin{array}[c]{r}
      1 \\ -1 \\ -1
    \end{array}
\right) \\[4ex]
%%%%%%
  \vec{d_{3}} = &
\frac{a_{\text{latt}}}{4}\left( 
    \begin{array}[c]{r}
      -1 \\ 1 \\ -1
    \end{array}
\right) &
%%%
  \vec{d_{4}} = &
\frac{a_{\text{latt}}}{4}\left( 
    \begin{array}[c]{r}
      -1 \\ -1 \\ 1
    \end{array}
\right)
\end{equationarray*}
%
zu finden. 

Um Zwei-Zentren-Integrale $(l\pri{l}\lambda)$ f"ur ungeordnetes \GaInP\ zu
erhalten, interpolieren wir die von Jancu \emph{et al.} \cite{jsbb:98} f"ur
GaP und InP angegebenen Werte linear
%
\begin{displaymath}
  (l\pri{l}\lambda)_{\GaInP} = x (l\pri{l}\lambda)_{\mathrm{GaP}} + (1-x) (l\pri{l}\lambda)_{\mathrm{InP}}
\end{displaymath}
%
und analog f"ur die \emph{on-site} Energien. F"ur $x=0.51$ liefern diese
Parameter aber eine zu gro"se Bandl"ucke von 2,28~eV, was ziemlich genau dem
Mittelwert der direkten Bandl"ucken von GaP und InP entspricht. Da
verschiedene Untersuchungen \cite{LB17a} darauf hinweisen, da"s das
Leitungsband f"ur das nicht-lineare Verhalten der direkten Bandl"ucke
verantwortlich ist, f"uhren wir einen empirischen \emph{bowing} Faktor
f"ur die vier $(ss\sigma)$-artigen Zwei-Zentren-Integrale ein
%
\begin{displaymath}
  (ss\sigma)_{\GaInP} = x (ss\sigma)_{\mathrm{GaP}} + (1-x)
  (ss\sigma)_{\mathrm{InP}} + b_{(ss\sigma)} x(1-x)
\end{displaymath}
%
mit $b_{(ss\sigma)} =
\frac{1}{2}((ss\sigma)_{\mathrm{GaP}}-(ss\sigma)_{\mathrm{InP}})$. 
Analog f"ur $s_{a}s_{c}^{\ast}\sigma$, $s_{a}^{\ast}s_{c}\sigma$ und
$s^{\ast}s^{\ast}\sigma$. Die so erhaltenen Parameter sind in der rechten
Spalte von Tab.~\ref{tab:tba-parameter} angegeben.

Experimentelle Werte f"ur die Bandl"ucke bei tiefen Temperaturen schwanken
zwischen 1,975~eV und 2,006~eV \cite{kipp:97,fgmz:98}. Die mit den hier
verwendeten Parametern berechnete Bandl"ucke liegt mit 1.991~eV in diesem
Bereich. Dabei wurde bereits eingerechnet, da"s die Bandl"ucke um ein drittel
der Spin-Bahn-Aufspaltung von etwa 100~meV reduziert wird, wenn die
Spin-Bahn-Wechselwirkung ber"ucksichtigt wird.

Die in Kap.~\ref{sec:basiswahl} dargestellten Bandstrukturen, sowie die in
Kap.~\ref{sec:betrag} verwendeten effektiven Massen erhalten wir durch eine
numerische Diagonalisierung dieses TBA-Hamilton-Operators mit den in der
rechten Spalte von Tab.~\ref{tab:tba-parameter} angegebenen Parametern.

%%%%%%%%%%%%%%%%%%%%%%

\section{Berechnung von Matrixelementen}
\label{sec:me-berechnung}

Bei der Berechnung von Matrixelementen mit Hilfe der LCAO-Methode gehen wir
wie folgt vor. Zun"achst berechnen wir die Eigenfunktionen des
TBA-Hamilton-Operators des ungeordneten Systems f"ur $\Gamma$- und L-Punkt.
Diese Wellenfunktionen sind f"ur eine Zinkblende-Einheitszelle definiert,
lassen sich aber nach dem in Kap.~\ref{sec:sp3} gezeigten Verfahren auf die
Einheitszelle des \CuPt-geordneten Materials erweitern.

Der TBA-Hamilton-Operators des geordneten Systems zeichnet sich dadurch aus,
da"s die beiden Kationen C1 und C2 verschieden sind, und damit auch die
Zwei-Zentren-Integrale sich unterscheiden. So gilt i.~A.\ $p_{c2}p_{a1}\pi
\neq p_{c1}p_{a2}\pi$ etc. Diesen Hamilton-Operator k"onnen wir in eine Summe
aus zwei Matrizen aufteilen. In der ersten Matrix \op{H_{0}^{\SSs \text{TBA}}}
setzen wir "uberall die \emph{on-site} Energien und Zwei-Zentren-Integrale des
ungeordneten Systems ein, die zweite Matrix \op{H_{1}^{\SSs \text{TBA}}}
enth"alt dann die Unterschiede zum tats"achlichen Wert f"ur ein geordnetes
System.

Die auf die Einheitszelle des \CuPt-geordneten Materials erweiterten
Eigenfunktionen des ungeordneten Systems sind Eigenfunktionen zu
\op{H_{0}^{\SSs \text{TBA}}}. Das Ordnungspotential ist in der TBA-Basis durch
\op{H_{1}^{\SSs \text{TBA}}} gegeben, und dessen Matrixelemente bez"uglich der
Zust"ande aus Kap.~\ref{sec:sp3} k"onnen wir berechnen. Eines davon ist
in Gl.~\eqref{eq:V35-tba} angegeben. Die Unterschiede der Form $\Delta
l_{c1}\pri{l}_{a2}\lambda$ k"onnen wir durch den Unterschied zwischen
den entsprechenden Zwei-Zentren-Integralen f"ur GaP und InP multipliziert mit
dem Ordnungsgrad modellieren. Die Schwierigkeit hier ist, da"s die
Zwei-Zentren-Integralen vom Abstand abh"angen, aber f"ur die Form dieser
Abstandsabh"angigkeit verschiedene Modelle existieren
\cite[p.~504ff]{jsbb:98, harr:80}. Je nach dem welche Abstandsabh"angigkeit
(und welche Parametrisierung) wir verwenden, erhalten wir unterschiedliche
Ergebnisse. 

Eine M"oglichkeit um zu testen, ob die LCAO-Methode "uberhaupt
geeignet ist solche Matrixelemente zu berechnen, w"aren
(GaAs)$_{n}$(AlAs)$_{n}$ "Ubergitter f"ur verschiedenes $n$. Auch hier wurden
Schwierigkeiten mit der Standard \kdotp-Theorie berichtet \cite{wozu:96}, die
mit der hier diskutierten Verallgemeinerung nicht auftreten sollten. Der
Vorteil bei diesem System w"are, da"s GaAs und AlAs fast die gleiche
Gitterkonstante haben, so da"s die Schwierigkeiten mit der
Abstandsabh"angigkeit der Zwei-Zentren-Integrale vermieden werden k"onnen,
wenn Potentialmatrixelemente "uber TBA berechnet werden




%%% Local Variables: 
%%% mode: latex
%%% TeX-master: "diplom"
%%% End: 







