%%% Einleitungskapitel f�r Diplomarbeit
%%% Time-stamp: <1999-03-04 03:05:50 ralf>


\chapter{Einleitung}
\label{cha:ein}

Bei der Beschreibung der Eigenschaften von Festk"orpern stehen wir vor der
Schwierigkeit, da"s es sich dabei um ein wechselwirkendes Vielteilchensystem
handelt. Oft wird daher die N"aherung verwendet, dieses Vielteilchenproblem
auf ein effektives Einteilchenproblem abzubilden und die Einteilchenzust"ande
zu betrachten, die sich daraus ergeben. In Halbleitern werden die wesentlichen
Eigenschaften von jenen effektiven Einteilchenzust"ande bestimmt, die in der
N"ahe der Bandkanten liegen. Die \kdotp-Theorie \cite{kane:66} beschreibt
diese Zust"ande gut, weshalb sie sehr erfolgreich bei der Erkl"arung vieler
Effekte in Halbleitern ist und breite Anwendung findet, insbesondere in
Verbindung mit der Envelopefunktionsn"aherung \cite{luko:55}.
 
Allerdings ergeben sich Schwierigkeiten, wenn das betrachtete System sich auf
einer der Gitterkonstante vergleichbaren L"angenskala periodisch
"andert. Beispiele f"ur Systeme in denen dies untersucht wurde sind
Halbleiterlegierungen in denen spontane Ordnung auftritt \cite{fwz:95} und
kurzperiodische "Ubergitter \cite{wozu:96}. Diese Schwierigkeiten k"onnen wir
so verstehen, da"s die \kdotp-Theorie Bereiche der Brillouin-Zone schlecht
beschreibt, die im reziproken Raum einen gro"sen Abstand von den
Bandkantenzust"anden haben. Doch es sind diese weit entfernten Zust"ande, die
in Systemen mit kurzperiodischer St"orung entscheidend beitragen. 

Aus diesem Grund konnten solche Systeme meist nur mit
All-Elektronen-Rechnungen behandelt werden. Die Ergebnisse solch einer
Rechnung sind aber schwieriger zu interpretieren, als Ergebnisse von
Rechnungen mit \kdotp-Theorie. Deshalb wollen wir hier zeigen, wie sich die
\kdotp-Theorie so erweitern l"a"st, da"s periodische St"orungen effektiv
beschrieben werden k"onnen.
Als Beispiel f"ur eine Anwendung dieser Verallgemeinerung soll die
Ordnungsabh"angigkeit der effektiven Masse im Leitungsband von spontan
geordnetem \GaInP\ untersucht werden.


\newpage
Diese Arbeit gliedert sich wie folgt. 

Im zweiten Kapitel werden wir die Standard \kdotp-Theorie
skizzieren und die notwendigen Verallgemeinerungen f"ur die Behandlung
periodischer St"orungen durchf"uhren.

Das dritte Kapitel enth"alt eine Einf"uhrung in wichtige Eigenschaften des
Materialsystems \GaInP. Danach zeigen wir mit welchem Hamilton-Operator die
effektive Masse des Leitungsbandes in diesem Material beschrieben werden kann.

Im vierten Kapitel l"osen wir diesen Hamilton-Operator und pr"asentieren die
Ergebnisse dieser Rechnung. 

Das letzte Kapitel enth"alt eine Zusammenfassung und einen kurzen Ausblick auf
k"unftige Weiterentwicklungen.

%%% Local Variables: 
%%% mode: latex
%%% TeX-master: "diplom"
%%% End: 
