\usepackage[osf,slantedGreek]{mathpazo}
\linespread{1.15}\selectfont
\typearea[8mm]{16}% calc gives 20
\addtokomafont{sectioning}{\rmfamily}
\addtokomafont{descriptionlabel}{\rmfamily}
\addtokomafont{pagehead}{\itshape}
\usepackage{bm}
\DeclareUnicodeCharacter{2082}{\textsubscript{2}}

\newcommand{\V}[1]{V_{#1}}
\newcommand{\kk}{\vec{k}}
\newcommand{\pp}{\vec{p}}
\newcommand{\rr}{\vec{r}}
\newcommand{\veps}{\varepsilon}
\newcommand{\kdotp}{\kk \cdot \pp}

\newcommand{\mathcalbf}[1]{\bm{\mathcal{#1}}}
% primed
\newcommand{\pri}[1]{{#1^{\prime}}}
% k in aBZ
\newcommand{\ak}{\grvec{\kappa}}
\newcommand{\akb}{{\kappa}}
% k in nBZ
\newcommand{\nk}{\vec{k}}
\newcommand{\nkb}{{k}}
% Set an k-punkten
%\newcommand{\set}{\vec{K}}
\newcommand{\set}{{\mathcalbf{K}}}
% reziproke Gittervektoren
\newcommand{\aG}{{\mathcalbf{G}_{\mathnormal{m}}}}
\newcommand{\paG}{{\mathcalbf{G}_{\mathnormal{\pri{m}}}}}
\newcommand{\nG}{{\vec{G}_{\mathnormal{m}}}}
\newcommand{\pnG}{{\vec{G}_{\mathnormal{\pri{m}}}}}
% Basisfunktionen
\newcommand{\altebasis}{{\phi_{n \ak}}}
\newcommand{\paltebasis}{{\phi_{\pri{n} \ak}}}
\newcommand{\basis}{{\chi^{\set}_{n\nk}}}
\newcommand{\varbasis}{{n\,\nk\,\set}}
\newcommand{\pbasis}{{\chi^{\pri{\set}}_{\pri{n}\pri{\nk}}}}
\newcommand{\pvarbasis}{{\pri{n}\,\pri{\nk}\,\pri{\set}}}
% Integral mit text drunter
\newcommand{\bzint}[1]{{\int\limits_{\scriptscriptstyle \text{#1}}}}
% Entwicklungskoeffizienten
%\newcommand{\BnnKK}[1]{{B_{\begin{array}{l}\Ss \!\! n\pri{n}\\\Ss \!\! \set\pri{\set}\end{array}}^{#1}}}
\newcommand{\BnnKK}[1]{{B_{\!\!\!{n\pri{n} \atop \set\pri{\set}}}^{#1}}}
\newcommand{\koeff}{{A^{\set}_{n} \! (\nk) \,}}
\newcommand{\pkoeff}{{A^{\pri{\set}}_{\pri{n}} \! (\pri{\nk}) \,}}
% Impuls- und Potential-Matrixelement
\newcommand{\pnn}{{\vec{p}_{\mathnormal{n\pri{n}}}}}
\newcommand{\pnnK}{{\vec{p}^{\set}_{\mathnormal{n\pri{n}}}}}
\newcommand{\VnnKK}{{V^{\set \pri{\set}}_{n \pri{n}}}}
% für 11, 21 und 22 Quadranten
\newcommand{\VnnKKee}{{V^{\set \set}_{n \pri{n}}}}
\newcommand{\VnnKKze}{{V^{\pri{\set} \set}_{n \pri{n}}}}
\newcommand{\VnnKKzz}{{V^{\pri{\set} \pri{\set}}_{n \pri{n}}}}
%
% mathematische Objekte
%
\newcommand{\Matrix}[1]{{#1}}
% Operator (ohne Hut)
\renewcommand{\vec}[1]{\bm{#1}}
\newcommand{\op}[1]{#1}
\newcommand{\vecop}[1]{\bm{#1}}
\newcommand{\grvec}[1]{\bm{#1}}
\newcommand{\grvecop}[1]{\bm{#1}}
%
\newcommand{\menge}[1]{\mathcal{#1}}
\newcommand{\raum}[1]{\mathsf{#1}}
% Einheitsvectoren
\newcommand{\ex}{\vec{\hat e}_{x}}
\newcommand{\ey}{\vec{\hat e}_{y}}
\newcommand{\ez}{\vec{\hat e}_{z}}
% hermitesch konjugiert, transponiert, komplex konjugiert
\newcommand{\hc}[1]{#1^{\dagger}}
\newcommand{\transp}[1]{#1^{\mathsf{T}}}
\newcommand{\cc}[1]{#1^{\ast}}
% Skalarprodukt <#1, #2> und vec.vec
\newcommand{\scp}[2]{\left\langle {#1}, {#2} \right\rangle}
\newcommand{\sprod}[2]{\vec{#1} \cdot \vec{#2}}
% Kommutator [#1, #2]
\newcommand{\kom}[2]{\left[\, {#1}, {#2} \, \right]}
% Antikommutator {#1, #2}
\newcommand{\akom}[2]{\left\{\, {#1}, {#2} \, \right\}}
% Ableitungen etc.
\newcommand{\dx}[1]{d#1\,}
\newcommand{\vdx}[1]{d^{3}\!\!\: #1\,}
\newcommand{\ddx}[1]{\frac{d}{d#1}}
\newcommand{\pdx}[1]{\frac{\partial}{\partial#1}}
% Benannte mathem. Objekte
\newcommand{\kronecker}[2]{{\delta_{#1\,#2}}}
% bra und ket etc...
\newcommand{\bra}[1]{{\langle {#1}|}}
\newcommand{\ket}[1]{{| {#1} \rangle}}
\newcommand{\braket}[2]{{\langle {#1} | {#2} \rangle}}
\newcommand{\matrixel}[3]{{\langle {#1} | {#2} | {#3} \rangle}}
\newcommand{\expect}[1]{{\langle {#1} \rangle}}
% für große Einträge
\newcommand{\Bra}[1]{{\left\langle {#1} \right|}}
\newcommand{\Ket}[1]{{\left| {#1} \right\rangle}}
\newcommand{\BraKet}[2]{{\left\langle {#1} | {#2} \right\rangle}}
\newcommand{\Matrixel}[3]{{\left\langle {#1}%
\left| {#2} \right| {#3} \right\rangle}}
\newcommand{\Expect}[1]{{\left\langle {#1} \right\rangle}}
